%\RequirePackage{scrpage2} %% obsolete --> use scrlayer-scrpage instead if needed
\RequirePackage[utf8]{inputenc}
\RequirePackage[T1]{fontenc}
\RequirePackage[ngerman]{babel}
\RequirePackage[german]{babelbib}
\RequirePackage{microtype}
\RequirePackage{pifont}
\RequirePackage{lmodern}
\RequirePackage{latexsym,amssymb,amsmath,mathdots}
\RequirePackage{wasysym}
% \RequirePackage{txfonts}
% \RequirePackage[scaled=.9]{helvet}
\RequirePackage{xspace}
\RequirePackage[pdfborder={0 0 0}]{hyperref}
\RequirePackage{graphicx}
\RequirePackage{tikz}
\usetikzlibrary{automata,positioning,arrows.meta,shapes,calc,fit}
\RequirePackage{ifthen}
\RequirePackage{rotating}

\RequirePackage{chngcntr}
\counterwithout{figure}{section}
\renewcommand\thefigure{\arabic{figure}}
\counterwithout{equation}{section}

%\DeclareFontShape{U}{wasy}{b}{n}{%
%<->wasyb10%
%}{}

\wasyfamily
\DeclareFontShape{U}{wasy}{b}{n}{ <-10> ssub * wasy/m/n
 <10> <10.95> <12> <14.4> <17.28> <20.74> <24.88>wasyb10 }{}

\renewcommand\thepart{\arabic{part}}
\renewcommand\partname{Kapitel}

\renewcommand\floatpagefraction{.99}
\renewcommand\topfraction{.99}
\renewcommand\bottomfraction{.99}
\renewcommand\textfraction{.01}

\pgfdeclarelayer{background}
\pgfdeclarelayer{foreground}
\pgfsetlayers{background,main,foreground}

\hyphenation{Konzept-inklusion}

\newcommand{\textbfsf}[1]{\textbf{\textsf{#1}}}
\newcommand{\textsfbf}[1]{\textbf{\textsf{#1}}}

% Beweisumgebung
\newcommand{\qedsymbol}{\ding{111}}
\newcommand{\qedhere}{\hspace*{\fill}\qedsymbol}
\newcommand{\quasiqedhere}{\hspace*{\fill}(\qedsymbol)}
\newenvironment{beweis}{%
  \par\medskip\noindent
  \textup{\textsfbf{Beweis.}}
}{%
%   \hspace*{\fill}\qedsymbol
  \par\smallskip\noindent
}

% ----- Frakturbuchstaben -----
\newcommand{\Amf}{\ensuremath{\mathfrak{A}}\xspace}
\newcommand{\Bmf}{\ensuremath{\mathfrak{B}}\xspace}
\newcommand{\Imf}{\ensuremath{\mathfrak{I}}\xspace}
\newcommand{\Nmf}{\ensuremath{\mathfrak{N}}\xspace}

% ----- kalligrafische Buchstaben -----
\newcommand{\Amc}{\ensuremath{\mathcal{A}}\xspace}
\newcommand{\Bmc}{\ensuremath{\mathcal{B}}\xspace}
\newcommand{\Fmc}{\ensuremath{\mathcal{F}}\xspace}
\newcommand{\Gmc}{\ensuremath{\mathcal{G}}\xspace}
\newcommand{\Imc}{\ensuremath{\mathcal{I}}\xspace}
\newcommand{\Jmc}{\ensuremath{\mathcal{J}}\xspace}
\newcommand{\Kmc}{\ensuremath{\mathcal{K}}\xspace}
\newcommand{\Lmc}{\ensuremath{\mathcal{L}}\xspace}
\newcommand{\Omc}{\ensuremath{\mathcal{O}}\xspace}
\newcommand{\Pmc}{\ensuremath{\mathcal{P}}\xspace}
\newcommand{\Smc}{\ensuremath{\mathcal{S}}\xspace}
\newcommand{\Tmc}{\ensuremath{\mathcal{T}}\xspace}
\newcommand{\Umc}{\ensuremath{\mathcal{U}}\xspace}

% ----- Carstens phi-Abkürzung -----
\newcommand{\vp}{\ensuremath{\varphi}\xspace}

% ----- Makros für lineare Strukturen und deren Labels -----
\newcommand{\linstruc}[1]{%
  \node[state] (elem0) {0};
  \foreach \x [remember=\x as \lastx (initially 0)] in {1,...,#1} {
      \node[state] (elem\x) [right=of elem\lastx] {\x};
      \path[->] (elem\lastx) edge node[above] {$s$} (elem\x);
  };

  \path[->] (elem#1) edge[loop right] node[right] {$s$} ();
}

% 1st param: location of label w.r.t. to element (default: above)
% 2nd param: distance to element
% 3rd param: element name
% 4th param: alignment among labels
% 5th param: label text (without "tabular")
\newcommand{\linstruclab}[5][above]{%
  \node [#1=#2 of elem#3] {\begin{tabular}{@{}#4@{}}#5\end{tabular}};
}

\renewcommand{\Vec}[1]{\ensuremath{\overrightarrow{#1}}}

\newcommand{\bspnr}[1]{{\footnotesize\textsf{#1}}}

\newcommand{\auf}{\ensuremath{\langle}}
\newcommand{\zu}{\ensuremath{\rangle}}

\newcommand*\circled[1]{\tikz[baseline=(char.base)]{
            \node[shape=circle,draw,inner sep=1pt,line width = .8pt] (char) {{\small #1}};}}

\newcommand{\term}[1]{\textsf{#1}}

\newcommand{\DL}[1]{\ensuremath{\mathcal{#1}}}
\newcommand{\ALC}{\DL{ALC}\xspace}
\newcommand{\ALCI}{\DL{ALCI}\xspace}
\newcommand{\ALCQ}{\DL{ALCQ}\xspace}
\newcommand{\ALCQI}{\DL{ALCQI}\xspace}

\newcommand{\qnrgeq}[3]{\ensuremath{(\mathord{\geqslant}#1\,#2.#3)}}
\newcommand{\qnrleq}[3]{\ensuremath{(\mathord{\leqslant}#1\,#2.#3)}}

\newcommand{\parI}{\par\smallskip}
\newcommand{\parII}{\par\medskip}
\newcommand{\parIII}{\par\bigskip}

\newenvironment{tboxarray}{%
  \renewcommand{\arraystretch}{1.2}
  \begin{array}{@{}r@{~~}r@{~~}c@{~~}l@{~~}l@{}}
}{%
  \end{array}
}